\documentclass{article}
\usepackage[utf8]{inputenc}

\title{Mjukvaruutveckling av visualiseringssystem}
\author{Rohullah Khorami \& Fredrik Kortetjärvi }


\begin{document}

\maketitle

\section{Introduktion}Det är ett stort behov av ett visualiseringssystem kopplat mot fastighetssystemet KNX. Att driftsätta och testa av
en större fastighet är mycket tidskrävande och risken för att missa någon avvikelse är stor. 
För systemet finns en färdig drivrutin för Windows och .Net, med funktioner för att koppla upp mot systemet.
Applikationen ska således utvecklas mot .Net i valfritt passande språk.

\section{Syfte}Den tilltänkta applikationen kommer kvalitetssäkra testning och minska tidsåtgången att testa systemet på olika fastigheter avsevärt.

\section{Metod}
Planen är att ta fram en applikation från grunden som kopplar upp mot
fastighetssystemet (via befintlig drivrutin) och lyssnar av trafiken för att logga och visa upp denna grafiskt.
Applikationen behöver vara dynamisk, dvs. kunna laddas med strukturerad data (t.ex. från xml eller csv) för att
med denna data automatiskt generera visning och kontroller i applikationen enligt uppsatta regler.\newline

Applikationen kommer vara i C\# för att kunna koppla applikationen med .NET drivrutin. Det behöver bara vara i Windows då företaget bara använder Windows. Andra föredel med C\# är att graphical user interface(GUI) delen är "inbyggd". Det går att använda andra språk också t.ex C eller C++ och dessa har fördelar som har en snabb exekveringstid och är lättviktigt språk. C och C++ är cross-plattform program fast C \& C++ har separata GUI bibliotek därför det är lämplig att använda C\# för programmera applikationen.\newline 
Under UtExpo kommer mjukvaran och relation mellan mjukvaran och hårdvaror visas. Det innebär att hur hårdvaror kommer kommunicera med applikationen och även koden kan presenteras om det behövs.

\section{Tidplan} 
\begin{itemize}
    \item Planering av GUI delen, hur applikationen kommer se ut grafiskt och det kommer ta ungefär två dagar.
    \item Delar applikationen i små delar som kan arbetas separat och det kommer ta ungefär 3 timmer. 
    \item Själva programmering processen kommer ta ungefär 8 veckor med återkoppling med handledarna varje vecka.
    \item På börja både presentation och rapport till seminarie tillfällena.Detta görs då och då under programmingstiden. 
    \item Efter 10 veckor på börjas rapport skrivningen till slutseminariet.
    \item Förberedelse till UtExpo en vecka innan UtExpo.
\end{itemize}
\end{document}
