\documentclass{article}
\usepackage[utf8]{inputenc}

\title{Mjukvaruutveckling av visualiseringssystem\\
\Large {Projektplan}
}

\author{Rohullah Khorami \& Fredrik Kortetjärvi }


\begin{document}

\maketitle
\newpage
\tableofcontents
\newpage

\section{Introduktion}Det är ett stort behov av ett visualiseringssystem kopplat mot fastighetssystemet KNX. Att driftsätta och testa av
en större fastighet är mycket tidskrävande och risken för att missa någon avvikelse är stor. 
För systemet finns en färdig drivrutin för Windows och .Net, med funktioner för att koppla upp mot systemet.
Applikationen ska således utvecklas mot .Net i valfritt passande språk.

\section{Syfte}Den tilltänkta applikationen kommer kvalitetssäkra testning och minska tidsåtgången att testa systemet på olika fastigheter avsevärt.

\section{Metod}
Planen är att ta fram en applikation från grunden som kopplar upp mot
fastighetssystemet (via befintlig drivrutin) och lyssnar av trafiken för att logga och visa upp denna grafiskt.
Applikationen behöver vara dynamisk, dvs. kunna laddas med strukturerad data (t.ex. från xml eller csv) för att
med denna data automatiskt generera visning och kontroller i applikationen enligt uppsatta regler.\newline

\subsection{Kunskapsläget}
Applikationen kommer vara i C\# för att kunna koppla applikationen med .NET drivrutin. Det behöver bara vara i Windows då företaget bara använder Windows. Andra föredel med C\# är att graphical user interface(GUI) delen är "inbyggd". Det går att använda andra språk också t.ex C eller C++ och dessa har fördelar som har en snabb exekveringstid och är lättviktigt språk. C och C++ är cross-plattform program fast C \& C++ har separata GUI bibliotek därför det är lämplig att använda C\# för programmera applikationen.\newline 
\subsection{Person \& Resurs}
Sundas Munir kommer vara vår handledare genom att hon hjälper med att skriva rapporten och diskutera kring problem. Marcus Karlsson är vår kontakt person på Teknisk Byrån som kommer ge oss mer information och resurser. Projekt processen har inte något kostnad för att företaget kommer fixa visa hårdvaror för att testa applikationen i slutet. Projektet kommer görs i Högskolan eller hemma då behöver vi inte träffa företaget då finns inget reskostnad heller. \newline
\subsection{Material}
Visual studio 2019 och C\# kommer används för att programmera applikationen för att det är enklare att lansera koden och har bra verktyg att debugger och integrerar med dll filen.\newline
\subsection{Resultat \& analys}
Programmet kommer testas med hjälp av hårdvara från företaget och kommer även testas av företaget själva för att applicera produkten i verkligheten. Efter testet kommer respons från företaget som avgör hur färdig produkten är samt hur effektivt och användbar programmet är i dags läget. Felen som kan kommas komma upp är att resultatet av programmet inte är användbart eller oläsligt. Efter responsen kommer dra slutsats hur mycket som måste göras för att bli färdiga med produkten.
\subsection{UtExpo}
Under UtExpo kommer mjukvaran och relation mellan mjukvaran och hårdvaror visas. Det innebär att hur hårdvaror kommer kommunicera med applikationen och även koden kan presenteras om det behövs.\newline
\newpage
\section{Tidplan} 
\begin{center}
    \begin{tabular}{|c|c|}
    \hline
        \textbf {Uppgift} &  \textbf{Datum} \\
        \hline
        Planering av GUI delen, hur applikationen kommer se ut grafiskt & 31-01-2021 \\
        \hline
        Delar applikationen i små delar som kan arbetas separat & 01-02-2021\\
        \hline
        Själva programmering processen & 01-02-2021\\
        \hline
        Halvtids seminarie färdigt & 12-03-2021\\
        \hline
        På börja både presentation  & 10-05-2021 \\
        \hline
        Färdig rapport & 14-05-2021\\
        \hline
        Förberedelse till UtExpo(om det blir av) & Uexpo en vecka innan \\
    \hline
    \end{tabular}
\end{center}
\end{document}
